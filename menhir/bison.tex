\section{Comparison Menhir/Bison}
\setcounter{subsection}{1}

\begin{lrbox}{\codebox}
  \begin{tcblisting}{listing only,title=Menhir: Basic arithmetics}
E:
| exp=E `+' term=T     { exp + term }
| exp=E `-' term=T     { exp - term }
| term=T               { term       }

T:
| term=T `*' factor=F  { term * factor }
| num=T `/' denom=F    { num / denom   }
| factor=F             { factor        }

F:
| n=NUMBER             { n    }
| `-' neg=F            { -neg }
| `(' exp=E `)'        { exp  }
  \end{tcblisting}
\end{lrbox}
\begin{frame}
\frametitle{Example Menhir grammar}
  \usebox{\codebox}
\end{frame}

\begin{frame}[fragile]
  \frametitle{Parametrized rules}
  Target language:
  \begin{verbatim}
enum numbers {
  one
  two
  three
}

enum forward;

if (true) { do; stuff; }

if (true);
\end{verbatim}
\end{frame}

\lstset{basicstyle=\ttfamily\scriptsize}
\begin{lrbox}{\codebox}
  \begin{tcolorbox}[title=\centering{Parametrized rules}\\Bison\hspace{0.45\linewidth} Menhir,sidebyside]%
  \begin{lstlisting}
enum:
  ENUM NAME
  `{' value_list `}'
| ENUM NAME `;'

if_block:
  IF `(' condition `)'
  `{' statement_list `}'
| IF `(' condition `)' `;'

value_list:
  %empty | value_list value

stmt_list:
  %empty | stmt_list stmt
\end{lstlisting}
\tcblower%
  \begin{lstlisting}
enum:
  ENUM NAME
  opt_block(list(value))

if_block:
  IF `(' condition `)'
  opt_block(list(stmt))

opt_block(INSIDE):
| `{' INSIDE `}'
| `;'
  \end{lstlisting}
  \end{tcolorbox}
\end{lrbox}
\begin{frame}
  \frametitle{Parametrized rules}
  \usebox{\codebox}
\end{frame}

\lstset{basicstyle=\ttfamily\tiny}
\begin{lrbox}{\codebox}
  \begin{tcolorbox}[title=\centering{Parametrized rules}\\Bison\hspace{0.45\linewidth} Menhir,sidebyside]%
  \begin{lstlisting}
enum:
  ENUM NAME `{' value_list `}'
  { $$ = new EnumNode($2, $4); }
| ENUM NAME `;'
  { $$ = new EnumNode($2, NULL); }

if_block:
  IF `(' condition `)'
  `{' statement_list `}'
  { $$ = new IfNode($3, $5); }
| IF `(' condition `)' `;'
  { $$ = new IfNode($3, Node()); }

value_list:
  %empty { $$ = new vector<Val>(); }
| value_list value
  { $1->append($2); $$ = $1; }

stmt_list:
  %empty { $$ = new vector<St>(); }
| stmt_list stmt
  { $1->append($2); $$ = $1; }
\end{lstlisting}
\tcblower%
  \begin{lstlisting}
enum:
  ENUM name=NAME
  block=opt_block(list(value))
  { EnumNode (name, block) }

if_block:
  IF `(' cond=condition `)'
  block=opt_block(list(stmt))
  { let inside = match block with
    | None -> EmptyNode
    | Some a -> Node a
    in IfNode inside }

opt_block(INSIDE):
| `{' res=INSIDE `}'
  { Some res }
| `;'    { None }
  \end{lstlisting}
  \end{tcolorbox}
\end{lrbox}
\begin{frame}
  \frametitle{Parametrized rules}
  \usebox{\codebox}
\end{frame}


\lstset{basicstyle=\ttfamily\small}
\begin{lrbox}{\codebox}
  \begin{tcolorbox}[title=\centering{\%inline}\\Bison\hspace{0.45\linewidth} Menhir,sidebyside]%
  \begin{lstlisting}
%left PLUS
%left TIMES
%%

exp:
  NUMBER
| exp PLUS exp
| exp TIMES exp
\end{lstlisting}
\tcblower%
  \begin{lstlisting}
%left PLUS
%left TIMES
%%

exp:
| NUMBER
| exp op exp

%inline op:
| PLUS
| TIMES
  \end{lstlisting}
  \end{tcolorbox}
\end{lrbox}

\begin{frame}
  \frametitle{\%inline}
  \usebox{\codebox}
\end{frame}


\begin{frame}[fragile]
  \frametitle{\%inline example}
  Target language:
  \begin{verbatim}
a : int; //simple var
b : int array; //array of int
c : 0 : int array; //default value
\end{verbatim}
\end{frame}

\lstset{basicstyle=\ttfamily\scriptsize}

\begin{lrbox}{\codebox}
  \begin{tcolorbox}[title=\centering{First shot}\\Bison\hspace{0.45\linewidth} Menhir,sidebyside]%
  \begin{lstlisting}
program:
  var | array

var: IDENT COLON TYPE
     SEMICOLON

array:
  IDENT opt_value COLON
  TYPE ARRAY SEMICOLON

opt_value:
|  COLON VALUE
\end{lstlisting}
\tcblower%
  \begin{lstlisting}
program:
| variable {}
| arr {}

variable:
| IDENT COLON TYPE
  SEMICOLON {}

arr:
| IDENT option(VALUE) COLON
  TYPE ARRAY SEMICOLON {}
  \end{lstlisting}
  \end{tcolorbox}
\end{lrbox}

\begin{frame}
  \frametitle{First shot}
  \usebox{\codebox}
\end{frame}

\begin{frame}[fragile]
  \frametitle{Conflict explanation: Bison}
  \begin{lstlisting}
Rules never reduced

    5 opt_value: /* empty */


State 1 conflicts: 1 shift/reduce

...


state 1

    3 var: IDENT . COLON TYPE SEMICOLON
    4 array: IDENT . opt_value COLON TYPE ARRAY SEMICOLON

    COLON  shift, and go to state 5

    COLON  [reduce using rule 5 (opt_value)]

    opt_value  go to state 6
    \end{lstlisting}
\end{frame}


\begin{frame}[fragile]
  \frametitle{Conflict explanation: Menhir}
  \begin{lstlisting}

** Conflict (shift/reduce) in state 1.
** Token involved: COLON
** This state is reached from program after reading:

IDENT

** The derivations that appear below have the following
** common factor:
** (The question mark symbol (?) represents the spot where
** the derivations begin to differ.)

program
(?)
    \end{lstlisting}
\end{frame}

\begin{frame}[fragile]
  \frametitle{Conflict explanation: Menhir}
  \begin{lstlisting}

** In state 1, looking ahead at COLON, reducing production
** option(VALUE) ->
** is permitted because of the following sub-derivation:

arr
IDENT option(VALUE) COLON TYPE ARRAY SEMICOLON
      .
// lookahead token appears

** In state 1, looking ahead at COLON, shifting is permitted
** because of the following sub-derivation:

variable
IDENT . COLON TYPE SEMICOLON
    \end{lstlisting}
\end{frame}

\begin{frame}[fragile]
  \frametitle{Conflict explanation: Menhir}
  \begin{lstlisting}
State 1:
arr -> IDENT . option(VALUE) COLON TYPE ARRAY SEMICOLON [ # ]
variable -> IDENT . COLON TYPE SEMICOLON [ # ]
-- On VALUE shift to state 2
-- On COLON shift to state 3
-- On option(VALUE) shift to state 6
-- On COLON reduce production option(VALUE) ->
** Conflict on COLON
    \end{lstlisting}
\end{frame}
\lstset{basicstyle=\ttfamily\scriptsize}
\begin{lrbox}{\codebox}
  \begin{tcolorbox}[title=\centering{\%inline}\\Bison\hspace{0.45\linewidth} Menhir,sidebyside]%
  \begin{lstlisting}
program:
  var | array

var: IDENT `:' TYPE `;'

array:
  IDENT `:' TYPE ARRAY `;'
| IDENT `:' VALUE `:'
  TYPE ARRAY `;'
\end{lstlisting}
\tcblower%
  \begin{lstlisting}
program:
  var | array

var: IDENT `:' TYPE `;'

array:
  IDENT opt_value `:'
  TYPE ARRAY `;'

%inline opt_value:
| (* empty *)
| `:' VALUE
  \end{lstlisting}
  \end{tcolorbox}
\end{lrbox}

\begin{frame}
  \frametitle{\%inline}
  \usebox{\codebox}
\end{frame}


\begin{lrbox}{\codebox}
  \begin{tcolorbox}[title=\centering{\%inline}\\Bison\hspace{0.45\linewidth} Menhir,sidebyside]%
  \begin{lstlisting}
program:
  var | array

var: IDENT `:' TYPE `;'

array:
  IDENT `:' TYPE ARRAY `;'
| IDENT `:' VALUE `:'
  TYPE ARRAY `;'
\end{lstlisting}
\tcblower%
  \begin{lstlisting}
program:
  var | array

var: IDENT `:' TYPE `;'

array:
  IDENT
  ioption(preceded(`:',
                   VALUE))
  `:' TYPE ARRAY `;'
  \end{lstlisting}
  \end{tcolorbox}
\end{lrbox}

\begin{frame}
  \frametitle{\%inline}
  \usebox{\codebox}
\end{frame}

\begin{frame}[fragile]
  \frametitle{BNF example}
  Target language:
  \begin{verbatim}
int a = { 2, 3, 5 }
const int b = { 42 }
\end{verbatim}
\end{frame}

\begin{lrbox}{\codebox}
  \begin{tcolorbox}[title=\centering{BNF}\\Bison\hspace{0.45\linewidth} Menhir,sidebyside]%
  \begin{lstlisting}
program:
  %empty
| program statement

statement:
  opt_const INT IDENT
  `=' `{' values `}'

opt_const:
  %empty
| CONST

values:
  NUMBER
| values NUMBER
\end{lstlisting}
\tcblower
  \end{tcolorbox}
\end{lrbox}

\begin{frame}
  \frametitle{BNF}
  \usebox{\codebox}
\end{frame}
\begin{lrbox}{\codebox}
  \begin{tcolorbox}[title=\centering{BNF}\\Bison\hspace{0.45\linewidth} Menhir,sidebyside]%
  \begin{lstlisting}
program:
  %empty
| program statement

statement:
  opt_const INT IDENT
  `=' `{' values `}'

opt_const:
  %empty
| CONST

values:
  NUMBER
| values NUMBER
\end{lstlisting}
\tcblower
  \begin{lstlisting}
program: statement*

statement:
  CONST? INT IDENT
  `=' `{' NUMBER+ `}'
  \end{lstlisting}
  \end{tcolorbox}
\end{lrbox}

\begin{frame}
  \frametitle{BNF}
  \usebox{\codebox}
\end{frame}


\begin{frame}[fragile]
  \frametitle{Split files advantage}
  \begin{itemize}
    \item Split the grammar into several files
    \item Modular, organized, not one huge file
    \item Separate headers from the grammar
    \item Several grammars for one set of tokens
  \end{itemize}
\end{frame}
