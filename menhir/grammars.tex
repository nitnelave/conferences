\section{Basics of grammars}
\setcounter{subsection}{1}


\begin{frame}
  \frametitle{What is a grammar?}
  \textbf{Grammar:} \textit{n.} "The study of how words and their component
  parts combine to form sentences."

  In computer science, the set of rules that describe a language.
\end{frame}


\begin{frame}
  \frametitle{What are words?}
  \begin{itemize}
    \item Text $\rightarrow$ tokens
    \item e.g. "function", '=', NUMBER
    \item It's the role of the lexer
  \end{itemize}
\end{frame}


\begin{frame}
  \frametitle{What are rules?}
  \begin{itemize}
    \item Terminals vs non-terminals
    \item Transform a series of symbols into a non-terminal
  \end{itemize}
\end{frame}

\begin{lrbox}{\codebox}
  \begin{tcblisting}{listing only,title=Yacc}
addition: NUMBER PLUS NUMBER

substraction: NUMBER MINUS NUMBER

operation:
  addition
| substraction
  \end{tcblisting}
\end{lrbox}
\begin{frame}
\frametitle{Yacc syntax}
  \begin{itemize}
    \item early 1970s
    \item Loosely based on BNF
  \end{itemize}
  \usebox{\codebox}
\end{frame}

\begin{frame}
  \frametitle{What is a parser?}
  \begin{itemize}
    \item Input: tokens
    \item "Groups" the token into rules
    \item Actions (e.g. AST creation)
  \end{itemize}
\end{frame}

\begin{lrbox}{\codebox}
  \begin{tcblisting}{listing only,title=Bison: Basic arithmetics}
E:
  E `+' T
| E `-' T
| T

T:
  T `*' F
| T `/' F
| F

F:
  NUMBER
| `-' F
| `(' E `)'
  \end{tcblisting}
\end{lrbox}
\begin{frame}
  \frametitle{Example grammar}
  \usebox{\codebox}
\end{frame}

\begin{lrbox}{\codebox}
  \begin{tcblisting}{listing only,title=Bison: Basic arithmetics}
E:
  E `+' T    { $$ = $1 + $3; }
| E `-' T    { $$ = $1 - $3; }
| T

T:
  T `*' F    { $$ = $1 * $3; }
| T `/' F    { $$ = $1 / $3; }
| F

F:
  NUMBER
| `-' F      { $$ = -$2;     }
| `(' E `)'  { $$ = $2;      }
  \end{tcblisting}
\end{lrbox}
\begin{frame}
  \frametitle{With actions}
  \usebox{\codebox}
\end{frame}


\begin{frame}
  \frametitle{Shift/reduce conflicts}
  \begin{itemize}
    \item Shift: add the symbol to the stack, not the end of a rule
    \item Reduce: the top of the stack matches a rule, pop it and add the rule
    production
    \item Conflict: both are possible
  \end{itemize}
\end{frame}

\begin{lrbox}{\codebox}
  \begin{tcblisting}{listing only,title=Bison: Dangling else}
statement:
  VALUE
| IF BOOL THEN statement
| IF BOOL THEN statement ELSE statement
  \end{tcblisting}
\end{lrbox}
\begin{frame}[fragile]
  \frametitle{Example: dangling else}
  \usebox{\codebox}
  Problem:

  \texttt{if a then \only<2->{(}if b then c\only<2>{)} else d\only<3>{)}}

\bigskip
  Whose \texttt{else} is it?
\end{frame}


\begin{frame}
  \frametitle{Shift/reduce conflict: details}
  \texttt{[> IF BOOL THEN IF BOOL THEN statement $\rightarrow$ ELSE <]}
  \begin{itemize}
    \item \texttt{IF BOOL THEN} is a valid rule -> reduce
    \item \texttt{ELSE} is a valid shift
    \item Conflict: both are possible
  \end{itemize}
\end{frame}
