\section{Logomatig}
\begin{frame}
\frametitle{What is Logomatig ?}
  \begin{itemize}
    \item \emph{Very} simple language
    \item Control a turtle
    \item Draw pretty things!
  \end{itemize}
\end{frame}

\begin{frame}
\frametitle{The language -- Basics}
  \begin{itemize}
    \item \texttt{forward \emph{n}}
    \item \texttt{right \emph{n}}
    \item \texttt{left \emph{n}}
  \end{itemize}
\end{frame}

\begin{frame}[fragile]
\frametitle{The language -- Basics}
\begin{verbatim}
forward 50
left 90
forward 20
\end{verbatim}

\begin{center}
\begin{tikzpicture}
  \draw (0, 0) -- (5, 0);
  \draw (5, 0) -- (5, 2);
\end{tikzpicture}
\end{center}
\end{frame}

\begin{frame}[fragile]
  \frametitle{The language -- \texttt{repeat}}
\begin{verbatim}
repeat 4 [
  forward 20
  left 90
]
\end{verbatim}

\begin{center}
\begin{tikzpicture}
  \draw (0, 0) -- (3, 0);
  \draw (3, 0) -- (3, 3);
  \draw (3, 3) -- (0, 3);
  \draw (0, 3) -- (0, 0);
\end{tikzpicture}
\end{center}
\end{frame}

\begin{frame}[fragile]
  \frametitle{The language -- \texttt{to} and \texttt{end}}
\begin{verbatim}
to line
  forward 20
  left 90
end
repeat 4 [
  line
]
\end{verbatim}

\begin{center}
\begin{tikzpicture}
  \draw (0, 0) -- (3, 0);
  \draw (3, 0) -- (3, 3);
  \draw (3, 3) -- (0, 3);
  \draw (0, 3) -- (0, 0);
\end{tikzpicture}
\end{center}
\end{frame}

